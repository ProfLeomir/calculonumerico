% Modelo de Aula em LaTeX
% IFRS - Ibirubá
% Se utilizar o Overleaf, clique em RICH TEXT acima para uma visualização mais suave do código!

\documentclass[a4paper,11pt]{article}

%-------------------------------------------------------------------------
% PACOTES (não é necessário alterar nada) 
%-------------------------------------------------------------------------

\usepackage[utf8]{inputenc}
\usepackage[brazil]{babel} 
\usepackage{graphicx}
\usepackage{amsfonts}
\usepackage{fancyhdr}
\usepackage{qrcode}
\usepackage{color} 
\usepackage{hyperref} 
\usepackage{geometry} 
\usepackage{pgfplots}
\usepackage{amsmath,amsthm,amsfonts,amssymb,amscd,amsxtra} 
\usepackage{multicol}
\usepackage{multirow}
\usepackage{enumerate}
\usepackage{tikz}
\usetikzlibrary{positioning,shapes,fit,arrows,through,calc,shapes.geometric,patterns,decorations.markings}
\usepackage{pgf,pgfplots}
\usepgfplotslibrary{fillbetween}
\pgfplotsset{compat=1.15}
\usepackage{mathrsfs}
\definecolor{myblue}{RGB}{56,94,141}
\definecolor{mygreen}{RGB}{51,153,0}
\definecolor{myred}{RGB}{204,0,0}
\usepackage{setspace}
\geometry{a4paper,left=1.5cm,right=1.5cm,top=1.5cm,bottom=1.5cm}
%%%%%%%%%%%%%%%%%%%Rodapé com o QR Code%%%%%%%%%%%%%%%%
\usepackage{atbegshi}
\AtBeginShipout{\AtBeginShipoutUpperLeft{%
  \put(\dimexpr\paperwidth-3cm\relax,-\dimexpr\paperheight-3cm\relax){%
    \begin{tikzpicture}[overlay]
      \qrcode[height=2cm]{https://cocalc.com/share/projects/9ef3ae7c-705e-41f9-b56a-4bf1c96e517c}
    \end{tikzpicture}%
  }%
}}
%-------------------------------------------------------------------------
% CABEÇALHO (não é necessário alterar nada) 
%-------------------------------------------------------------------------

\begin{document}

\pagestyle{empty}

\begin{minipage}{6cm} %Logo do IFPR
\resizebox{6cm}{1.6cm}{
\begin{tikzpicture}[square/.style={regular polygon,regular polygon sides=4}]
\tikzset{mynode/.style={square,rounded corners,draw=mygreen, top color=mygreen, bottom color=mygreen,very thick, text centered},}  
\node[circle, draw=myred , top color=myred, bottom color=myred] (q1) {$\quad$};
\node[mynode, below=0.08cm of q1] (q2) {$\;\;$};
\node[mynode, below=0.08cm of q2] (q3) {$\;\;$};
\node[mynode, below=0.08cm of q3] (q4) {$\;\;$};
\node[mynode, right=0.08cm of q1] (q5) {$\;\;$};
\node[mynode, below=0.08cm of q5] (q6) {$\;\;$};
\node[mynode, below=0.08cm of q6] (q7) {$\;\;$};
\node[mynode, below=0.08cm of q7] (q8) {$\;\;$};
\node[mynode, right=0.08cm of q5] (q9) {$\;\;$};
\node[mynode, right=0.08cm of q7] (q10) {$\;\;$};
\node[right=0.08cm of q8] (aux) {};
\node[text centered, right=0.06cm of q10] (q11) {\huge \sf \textbf{INSTITUTO FEDERAL}};
\node[mygreen, text centered, right=0.5cm of aux] (q12) {\huge\textsf{Rio Grande do Sul}};
\end{tikzpicture}}
\end{minipage}
\begin{minipage}{7cm}
\begin{center}
\begin{footnotesize}
{\Large \textbf{Trabalho II}}\\
\end{footnotesize}
\end{center}
\end{minipage}
\begin{minipage}{4cm}
\centering
\includegraphics[scale=0.12]{meBrasil.png}\\
\begin{scriptsize}
\textsf{Ministério da Educação}
\end{scriptsize}
\end{minipage}
\doublespacing
\begin{center}

\textbf{Ciências da Computação}  \ $-$ \ \textbf{ Valor: 2,0}  \\

\textbf{Cálculo I}  \ $-$ \ \textbf{Professor Me. Leomir A. S. Grave}   \\

\end{center}


\textbf{Nome:} \rule{12cm}{0.2mm} Data: 23/06/2023

\textit{Justifique suas respostas com cálculos.}
%-------------------------------------------------------------------------
% Escrever a Aula apartir daqui:
%-------------------------------------------------------------------------

\textbf{Questão 1} $-$ \textit{Peso: 0,25}

Calcule $\displaystyle\lim_{x\rightarrow 3}\dfrac{x^2-9}{(x-3)^2}$

\textbf{Questão 2} $-$ \textit{Peso: 0,5}

Calculando $\displaystyle\lim_{x\rightarrow 5^+}\dfrac{x^2+6}{(5-x)^2}$ e $\displaystyle\lim_{x\rightarrow 5^-}\dfrac{x^2+6}{(5-x)^2}$, podemos afirmar que $\displaystyle\lim_{x\rightarrow 5}\dfrac{x^2+6}{(5-x)^2}$ é $\infty$? \textit{(Demonstre através dos limites laterais)}.

\textbf{Questão 3} $-$ \textit{Peso: 0,25}

Encontre $\displaystyle\lim_{x\rightarrow +\infty}x^n+2n$ com $n \ \in \ \mathbb{N}^*$


\textbf{Questão 4} $-$ \textit{Peso: 0,5}

Calcule $\displaystyle\lim_{x\rightarrow \frac{\pi}{2}}\dfrac{\sqrt{1+\cos x}-\sqrt{1-\cos x}}{x}$


\textbf{Questão 5} $-$ \textit{Peso: 0,5}

Considere a função $f(x)=\left\{\begin{array}{lllll}x^2-1, \ \ \text{se} \ \ -1\leq x<0\\ 2x,  \ \ \text{se} \ \ 0\leq x<1 \\ 1,   \ \ \text{se} \ \ x = 1 \\ -2x+4, \ \ \text{se} \ \ 1< x\leq 2\\  0, \ \ \text{se} \ \  x>2 \end{array}$ , responda:\\

\begin{tikzpicture}
\begin{axis}[
  axis lines=middle,
  xlabel=$x$,
  ylabel=$f(x)$,
  xmin=-2,
  xmax=3,
  ymin=-2,
  ymax=5,
  xtick={-1,0,1,2},
  ytick={-1,0,1,2,3,4},
  legend pos=north west,
  legend cell align={left},
  width=8cm,
  height=6cm,
]

% Função -1 <= x < 0
\addplot[domain=-1:0, samples=100, color=black] {x^2 - 1};
%\addlegendentry{$f(x) = x^2 - 1$}

% Função 0 <= x < 1
\addplot[domain=0:1, samples=100, color=black] {2*x};
%\addlegendentry{$f(x) = 2x$}

% Ponto x = 1
\addplot[mark=*] coordinates {(1,1)};
%\addlegendentry{$(1, 1)$}

% Função 1 < x <= 2
\addplot[domain=1:2, samples=100, color=black] {-2*x + 4};
%\addlegendentry{$f(x) = -2x + 4$}

% Função x > 2
\addplot[domain=2:3, samples=100, color=black] {0};
%\addlegendentry{$f(x) = 0$}

% Bolinhas
\addplot[color=black, mark=*] coordinates {(-1,0)};
\addplot[color=black, mark=o] coordinates {(0,-1)};
\addplot[color=black, mark=*] coordinates {(0,0)};
\addplot[color=black, mark=o] coordinates {(1,1)};
\addplot[color=black, mark=*] coordinates {(1,1)};
\addplot[color=black, mark=*] coordinates {(2,0)};
\addplot[color=black, mark=o] coordinates {(1,2)};
\end{axis}
\end{tikzpicture}

a) Existe $f (0)$? 

b) Existe $\displaystyle \lim_{x \rightarrow 0^+ }f(x)$?

c) Existe $\displaystyle \lim_{x \rightarrow 0^- }f(x)$?

d) Existe $\displaystyle \lim_{x \rightarrow 0}f(x)$?

e) $f(x)$ é continua em $x=0$? 

 






\end{document}









%Fonte: 

